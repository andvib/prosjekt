\documentclass{article}
\usepackage[utf8]{inputenc}
\usepackage{amsmath, amssymb, bm}

\setlength\parindent{0pt}

\begin{document}
\subsection*{UAV Kinematics}
Position of the UAV will be given in reference frame \{n\} using NED. This gives the states:

\begin{equation}
	\bm{p}^n_{b/n} =
	\begin{bmatrix}
		N \\ E \\ D
	\end{bmatrix}
	=
	\begin{bmatrix}
		x_n \\ y_n \\ z_n
	\end{bmatrix}
\end{equation}

Attitude will be given as Euler-angles:

\begin{equation}
	\bm{\Theta}_{nb} = 
	\begin{bmatrix}
		\phi \\ \theta \\ \psi
	\end{bmatrix}
\end{equation}

Together these make up the position and orientation vector $\bm{\eta}$:

\begin{equation}
	\bm{\eta} = 
	\begin{bmatrix}
		\bm{p}^n_{b/n} \\ \bm{\Theta}_{nb}
	\end{bmatrix}
\end{equation}

*ADD 'THE WIND TRIANGLE'*

\subsection*{Camera position}

When assuming flat earth, the centre point of the camera on the ground can be expressed in the body frame \{b\} using the attitude $\bm{\Theta}_{nb}$ of the UAV and the height $z_n$:

\begin{equation}
	\bm{c}_b^b = 
	\begin{bmatrix}
		c_{x/b}^b \\ c_{y/b}^b
	\end{bmatrix}
	=
	\begin{bmatrix}
		z_n sin(\phi) \\ z_n sin(\theta)
	\end{bmatrix}
\end{equation}

The distance from the UAV to the camera centre point can be expressed in \{n\} by using the rotational matrix $\bm{R}_{z,\psi}$:

\begin{equation}
	\bm{c}^n_b = 
	\begin{bmatrix}
		c^n_{x/b} \\ c^n_{y/b}
	\end{bmatrix}
	= \bm{R}_{z, \psi}\bm{c}^b_b
\end{equation}

In order to translate this to position in NED, it needs to be added to the UAV's NED position:

\begin{equation}
	\bm{c}^n =
	\begin{bmatrix}
		x_n + c^n_{x/b} \\
		y_n + c^n_{y/b}
	\end{bmatrix}
\end{equation}

*DECIDE ON HAVING ONE CENTER POINT, OR TWO EXTREMITIES*


\end{document}