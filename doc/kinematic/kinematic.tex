\documentclass{article}
\usepackage[utf8]{inputenc}
\usepackage{amsmath, amssymb, bm}

\setlength\parindent{0pt}

\begin{document}

\subsection*{UAV Kinematics}
Position of the UAV will be given in reference frame \{n\} using NED. This gives the states:

\begin{equation}
	\bm{p}^n_{b/n} =
	\begin{bmatrix}
		N \\ E \\ D
	\end{bmatrix}
	=
	\begin{bmatrix}
		x_n \\ y_n \\ z_n
	\end{bmatrix}
\end{equation}

Attitude will be given as Euler-angles:

\begin{equation}
	\bm{\Theta}_{nb} = 
	\begin{bmatrix}
		\phi \\ \theta \\ \psi
	\end{bmatrix}
\end{equation}

Together these make up the position and orientation vector $\bm{\eta}$:

\begin{equation}
	\bm{\eta} = 
	\begin{bmatrix}
		\bm{p}^n_{b/n} \\ \bm{\Theta}_{nb}
	\end{bmatrix}
\end{equation}

\subsubsection*{Wind}
Wind will introduce what is called crab angle $\chi_c$ in the horizontal plane and the angle of attack $\gamma_a$ in the vertical plane. This will change the UAVs actual heading $\psi$ and pitch $\theta$ to:

\begin{subequations}
\begin{equation}
	\chi = \psi + \chi_c
\end{equation}
\begin{equation}
	\gamma = \theta + \gamma_a
\end{equation}
\end{subequations}
	
where $\chi$ is called the course and $\gamma$ is maybe called something???? These angles will only affect the navigational part, and not what the camera is pointed at.

\subsection*{Camera position}
When assuming flat earth, the centre point of the camera on the ground can be expressed in the body frame \{b\} using the attitude $\bm{\Theta}_{nb}$ of the UAV and the height $z_n$:

\begin{equation} \label{eq:camera_body_body}
	\bm{c}_b^b = 
	\begin{bmatrix}
		c_{x/b}^b \\ c_{y/b}^b
	\end{bmatrix}
	=
	\begin{bmatrix}
		z_n sin(\theta) \\ z_n sin(\phi)
	\end{bmatrix}
\end{equation}

The distance from the UAV to the camera centre point can be expressed in \{n\} by using the rotational matrix $\bm{R}_{z,\psi}$:

\begin{equation} \label{eq:body_ned_rotate}
	\bm{c}^n_b = 
	\begin{bmatrix}
		c^n_{x/b} \\ c^n_{y/b}
	\end{bmatrix}
	= \bm{R}_{z, \psi}\bm{c}^b_b
\end{equation}

In order to translate this to position in NED, it needs to be added to the UAV's NED position:

\begin{equation} \label{eq:body_ned_trans}
	\bm{c}^n =
	\begin{bmatrix}
		c_x^n \\ c_y^n
	\end{bmatrix}
	=
	\begin{bmatrix}
		x_n + c^n_{x/b} \\
		y_n + c^n_{y/b}
	\end{bmatrix}
\end{equation}

\subsubsection*{Camera Angle of View}
Since the camera isn't only focusing on one specific point, it can be useful describing the camera point of focus as two extremities instead of one center point. Equation \eqref{eq:camera_body_body} can easily be changed to do this. Assuming the camera as an angle of view $\sigma$, the equation now becomes

\begin{equation}
	\bm{e}_b^b = 
	\begin{bmatrix}
		e^b_{x_1/b} \\ e^b_{y_1/b} \\ e^b_{x_2/b} \\ e^b_{y_2/b}
	\end{bmatrix}
	=
	\begin{bmatrix}
		z_n sin(\theta + \sigma) \\
		z_n sin(\theta - \sigma) \\
		z_n sin(\phi + \sigma) \\
		z_n sin(\phi - \sigma)
	\end{bmatrix}
\end{equation}

The steps for translating the points to the NED frame are the same as in \eqref{eq:body_ned_rotate} and \eqref{eq:body_ned_trans}:

\begin{equation}
	\bm{e}_b^n =
	\begin{bmatrix}
		e^n_{x_1/b} \\ e^n_{y_1/b} \\ e^n_{x_2/b} \\ e^n_{y_2/b}
	\end{bmatrix}
	= \bm{R}_{z,\psi} \bm{e}_b^b
\end{equation}

\begin{equation}
	\bm{e}^n =
	\begin{bmatrix}
		e^n_{x_1} \\ e^n_{y_1} \\ e^n_{x_2} \\ e^n_{y_2}
	\end{bmatrix}
	=
	\begin{bmatrix}
		x_n + e^n_{x_1/b} \\
		y_n + e^n_{y_1/b} \\
		x_n + e^n_{x_2/b} \\
		y_n + e^n_{y_2/b}
	\end{bmatrix}
\end{equation}
	
	

*ADD TRANSLATIONAL MATRIX FOR POSITION OF CAMERA* \\

\end{document}