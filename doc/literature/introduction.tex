\section{Introduction}

Unmanned Aerial Vehicles (UAV) are widely used in ground observation by equipping the UAV with different kind of sensors. While the use of UAV eases many cases of ground observation, there are some difficulties related to the attitude of the aircraft. A camera that is fixed to the UAV will be coupled with the UAVs states, so that any change in attitude will cause the camera view to shift away from the points of interest. As the height increases the error increases, which means that even small attitude changes will give a difference in what is intended to be observed, and what is actually observed by the camera.

Today, it is common to attach a gimbal with a camera to the UAV to decouple the attitude of the UAV from the camera. This way the attitude of the UAV will not cause any errors in the image so that the operator can focus solely on the operation of the aircraft. However, the fuel costs of an aircraft with a gimbal attached may increase because of added weight and less effective aerodynamics.

This paper will investigate methods to reduce image errors caused by the UAVs attitude, while also avoiding the extra costs associated with a gimbal. The control methods will be developed with the usage of a pushbroom hyperspectral camera that is fixed to the UAV in mind. The alternative flight control methods that aim to decrease the errors caused by the coupling between the camera and the attitude of the UAV will be simulated, and their effects on the image error will be tested.