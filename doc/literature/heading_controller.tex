\subsection{Heading controller}
Heading is normally controlled by using the ailerons to roll the aircraft, with the resulting difference between the lift vectors of each wing causing the aircraft to turn. This strategy is used because it gives because it's most common in larger, manned aircrafts as it is causes little drag and is most comfortable for the passengers \cite{skidToTurnMills}. The main reason for why the same strategy is used with UAVs is that it is a familiar scheme for pilots and it is a well-tested way of heading control.

When using UAVs for ground operations the roll used to turn the aircraft is a big problem which is mostly avoided by attaching the camera to a gimbal that is counteracting the effects of the roll. When the UAV is not equipped with a gimbal different control strategies can be used to reduce the roll needed to turn or to ensure that the camera stays focused on the object. There is a benefit to creating new controllers for this as controllers allows for using existing trajectory planners.


\subsubsection{Rudder as a heading control surface}
A common method to avoid roll in UAV operation is to use the rudder to turn. The rudder is mostly used to reduce the sideslip angle of the aircraft. However, the rudder can be used to introduce sideslip which will cause the aircraft to turn. Common for these controllers is that they still use the ailerons to keep the wings level during flight.

%% RATC, FISHER %%
Controllers using rudder to turn the aircraft is sometimes referred to as Rudder Augmented Trajectory Correction (RATC) \cite{ratcFISHER}. In this paper a RATC controller is compared to a Aileron Only Trajectory Correction (AOTC) controller with focus on how they affect the resulting image error when using a camera fixed to the aircraft. The controller is implemented as a PD-controller simulated on a model of the Aerosonde UAV.

When simulating the image error is modeled with two terms. The first term is the lateral image error which comes from the aircraft having a lateral error in its flightline so that it is not positioned directly over the intended path, which leads to image error. The second term is the error that comes from banking the aircraft. It is modeled using simple trigonometry, and it is worth noting that this error increases as the altitude above ground is increased.

The simulation was done with two test cases, one without wind and one with wind, and the results for both of the cases was similar. The course error of the two controllers was very similiar, and unsurprisingly the AOTC controller had much more changes in roll and the RATC controller had much more changes in sideslip. The biggest difference was that the RATC controller used much more input to its control surfaces, up to $400 \%$ more than the AOTC controller.

When comparing the image error for the two controllers there was a big difference in performance. The RATC had very small errors while the AOTC controller had RMS errors over $300 m$ while the RATC stayed at about $20 m$, which shows that the RATC controller is a good choice for reducing image error. The control algorithms was also field tested, with results that matches the simulation results.

It is worth noting from this paper that successive loop closure is not needed to implement the RATC. This is because the control design only has a singel transfer function between desired heading to control surface deflection. Since AOTC requires successive loop closure the AOTC controller will have a slower response compared to RATC.

%% Heading Control of a Fixed Wing UAV Using Alternate Control Surfaces %%
A similar approach was taken in \cite{alternateSurfaceAhsan}, but a PID controller was used instead of a PD. The controller was created from a nonlinear model which have been linearized about a stable trim point, and the resulting rudder controller was compared with a controller using the aileron for heading control. The aileron controller was used in both cases to keep the wings level during the flight.

The simulations in this paper also shows that when using rudder as a control surface the aircraft has better response compared to aileron, with less overshoot and a lower steady state error. Bode plots of the two controllers show that the rudder based heading controller has a gain margin of $-24.5 dB$ and a phase margin of $87.1 \degree$, while the aileron based controller has a gain margin of $-25.7 dB$ and a phase margin of $94 \degree$. This means that the two controllers have pretty similar stability features.

%% Skid-to-Turn, MILLS %%
In \cite{skidToTurnMills} a rudder based controller is used to control an UAV that uses a camera to survey a locally linear infrastructure. In the paper the way of turning the aircraft is called skid-to-turn, and essentially does the same as the other controllers. To control the aircraft based on images a variation of controller called Image-Based Visual Servoing (IBVS) is used. This method identifies the structure that is to be surveyed, and creates a model of it as a straight line. This line becomes the track the UAV is supposed to follow, and the UAV will seek to minimize the track error. If required by the steady-state errors the wings may not be levelled and the aircrafts heading may not follow the track, as long as the line is within the camera's field of view (FOV). A PID controller is used to control the heading using the rudder.

The controller was simulated compared to a controller that banks the UAV to turn, and the results matches the previous results. Even though the bank-to-turn controller reduces the track error faster than skid-to-turn, skid-to-turn causes much less error in the image plane. Even though the camera's FOV at all times covered the structure being surveyed, the roll the UAV makes back and forth changes so quick that the images retrieve might very well be to blurry to be usable. The controllers were also tested in wind with similar results. One thing worth noticing is that when the skid-to-turn controller were to intercept the structure with tailwind it resulted in a significant overshoot. In the image plane however, the error was much smaller than for the bank-to-turn controller.


\subsubsection{Heading controller with constraints}
Since banking the aircraft quickly causes the camera's FOV to be away from our point of interest, it is possible to put constraints on the controller to ensure the camera stays focused on what we want. In \cite{constraintsEGBERT} this is done by putting constraints on the UAV roll and the above ground level (AGL) to track a roadway. The constraints are calculated from the camera's horizontal field of view, the assumed road width, and the expected turn angle of the road. In addition the AGL will influence the constraint for roll since these are highly connected.

The architecture of this system includes a 'Constraints Governor' that receives input from an image processing unit about the road it is following and telemetry input from the UAV about its position and heading. Based on these inputs the constraints ared calculated so that the road will stay within the camera's FOV. These constraints was forwarded to a previously made controller.

The system was simulated and only tested without wind. The simulated UAV successfully followed two $90 \degree$ turns with only losing the road from the camera's FOV two times. This was because the system did not estimate the road path well enough, and the paper argues that by pointing the camera forward the stiamtion can be improved.


\subsubsection{Conclusion}
Turning using rudder is the most common way to reduce the roll during UAV operations, and it gives a significant reduction in the image error because of this. These papers also show that even though it does not perform as good as the traditional controllers using ailerons when it comes to position and path following, the rudder controllers perform better in the image plane despite the slower convergence rate.

On interesting point made in \cite{ratcFISHER} is that the RATC controller will ease the flight plans for ground observing. When using AOTC controllers for ground observing extra measures often has to be taken to ensure that the entire area of interest is covered by the camera. For a typical $90 \degree$ turn this could be to fly past the turn, make complete circle in the opposite direction of the turn, and then continue on the path after then $90 \degree$ bend. When using the RATC controller developed, the flight path length and time was reduced by about $80 \%$, and the energy spent flying the corner was estimated to give an $75 \%$ redutcion in energy spent. This means that even though the paper concluded the RATC used $400 \%$ more input than the AOTC, the RATC will save time and maybe energy for complicated paths with many turns.