\subsection{Course controller}
The course of the aircraft is normally controlled by using the ailerons to roll the aircraft, with the resulting difference between the lift vectors of each wing causing the aircraft to turn. This strategy is the most common in larger, manned aircrafts as it is causes little drag and it is comfortable for the passengers \cite{skidToTurnMills}.

Banking to change the course of the aircraft leads to problems when performing ground observation, which is solved by decoupling the roll from the sensors by using a gimbal. In order to avoid the extra payload that the gimbal is, there exists different control strategies to reduce roll during course change.

*CLEAN UP TERMINOLOGY REGARDING HEADING/COURSE*


\subsubsection{Rudder as a course control surface}
While the rudder is most commonly used to reduce the sideslip during flight, it can also be used to create sideslip wich causes the aircraft to turn. The control method is a fairly common method to avoid roll during course change. KILDE HADDE VÆRT FINT. Common for these controllers is that they use the ailerons too keep the wings level during flight.

%% RATC, FISHER %%
A controller using this control strategy has been created by Thomas Fisher in his paper "Rudder Augmented Trajectory Correction for
Unmanned Aerial Vehicles to Decrease Lateral Image Errors of Fixed Camera Payloads" \cite{ratcFISHER}. Here the term 'Rudder Augmented Trajectory Correction' (RATC) is used for a controller using the rudder to change the course, and 'Aileron Only Trajectory Correction' (AOTC) for controllers using the ailerons as the course control surface. The implemented controller was a PD-controller simulated on a model of the Aerosonde UAV, and results focuses on image error when using a fixed camera.

%%When simulating the image error is modeled with two terms. The first term is the lateral image error which comes from the aircraft having a lateral error in its flightline so that it is not positioned directly over the intended path, which leads to image error. The second term is the error that comes from banking the aircraft. It is modeled using simple trigonometry, and it is worth noting that this error increases as the altitude above ground is increased.

The simulations showed that the course error for the two controllers were matching, both with and without wind. An unsurprisingly, the results show that the AOTC controller had much more changes in roll and the RATC controller had much more changes in sideslip. The biggest difference was that the RATC controller used much more input to its control surfaces, up to $400 \%$ more than the AOTC controller. BETTER WORD FOR "MORE CHANGE"?

When comparing the image error for the two controllers there was a big difference in performance. The image error was measured as the distance from the camera centre point to the desired ground path, and while image errors for the RATC controller stayed at about $20$ m the AOTC had a RMS error over $300$ m. Field tests show the same results and prove that RATC is a good choice for reducing image erros.

%It is worth noting from this paper that successive loop closure is not needed to implement the RATC. This is because the control design only has a singel transfer function between desired heading to control surface deflection. Since AOTC requires successive loop closure the AOTC controller will have a slower response compared to RATC.

%% Heading Control of a Fixed Wing UAV Using Alternate Control Surfaces %%
A similar approach was taken by Ahsan, Rafique and Abbas in \cite{alternateSurfaceAhsan}, but a PID controller was used instead of a PD. The controller was created from a nonlinear model which have been linearized about a stable trim point, and the resulting rudder controller was compared with a controller using the aileron for heading control.

The simulations in this paper also shows that when using rudder as a control surface the aircraft has better response compared to aileron, with less overshoot and a lower steady state error. Bode plots of the two controllers show that the rudder based course controller has a gain margin of $-24.5 dB$ and a phase margin of $87.1 \degree$, while the aileron based controller has a gain margin of $-25.7 dB$ and a phase margin of $94 \degree$. This means that the two controllers have similar stability features. THIS PARAGRAPH COULD BE BETTER

%% Skid-to-Turn, MILLS %%
Mills, Ford and Mejias refers to a rudder-based course controller as a 'skid-to-turn' controller in \cite{skidToTurnMills}, and in the paper it is used to control a UAV that uses a camera to survey a linear infrastructure. To control the aircraft based on images a variation of controller called Image-Based Visual Servoing (IBVS) is used. This method identifies the structure that is to be surveyed, and creates a model of it as a straight line. This line becomes the track the UAV is supposed to follow, and the UAV will seek to minimize the track error. If required by the steady-state errors the wings may not be levelled and the aircrafts heading may not follow the track, as long as the line is within the camera's field of view (FOV). A PID controller is used to control the heading using the rudder.

The controller was simulated compared to a controller that banks the UAV to turn, and the results matches the previous results. Even though the bank-to-turn controller reduces the track error faster than skid-to-turn, skid-to-turn causes much less error in the image plane. Even though the camera's FOV at all times covered the structure being surveyed, the roll the UAV makes back and forth changes so quick that the images retrieved might very well be to blurry to be usable. The controllers were also tested in wind with similar results. One thing worth noticing is that when the skid-to-turn controller were to intercept the structure with tailwind it resulted in a significant overshoot. In the image plane however, the error was much smaller than for the bank-to-turn controller.


\subsubsection{Heading controller with constraints}
Since banking the aircraft quickly causes the camera's FOV to be away from our point of interest, it is possible to put constraints on the banking angle to ensure the camera stays focused on what we want. In \cite{constraintsEGBERT} this is done by putting constraints on the UAV roll and the above ground level (AGL) to track a roadway. The constraints are calculated from the camera's horizontal field of view, the assumed road width, and the expected turn angle of the road. In addition the AGL will influence the constraint for roll since these are highly connected.

The architecture of this system includes a 'Constraints Governor' that receives input from an image processing unit about the road it is following and telemetry input from the UAV about its position and heading. Based on these inputs the constraints ared calculated so that the road will stay within the camera's FOV. These constraints was forwarded to a previously made controller.

The system was simulated and only tested without wind. The simulated UAV successfully followed two $90 \degree$ turns with only losing the road from the camera's FOV two times. This was because the system did not estimate the road path well enough, and the paper argues that by pointing the camera forward the estimation can be improved.


\subsubsection{Summary of review}
Turning using rudder is the most common way to reduce the roll during UAV operations, and it gives a significant reduction in the image error because of this. These papers also show that even though it does not perform as good as the traditional controllers using ailerons when it comes to position and path following, the rudder controllers perform better in the image plane despite the slower convergence rate.

One interesting point made in \cite{ratcFISHER} is that the RATC controller will ease the flight plans for ground observing. When using AOTC controllers for ground observing extra measures often has to be taken to ensure that the entire area of interest is covered by the camera. For a typical $90 \degree$ turn this could be to fly past the turn, make complete circle in the opposite direction of the turn, and then continue on the path after then $90 \degree$ bend. When using the RATC controller developed, the flight path length and time was reduced by about $80 \%$, and the energy spent flying the corner was estimated to give an $75 \%$ redutcion in energy spent. This means that even though the paper concluded the RATC used $400 \%$ more input than the AOTC, the RATC will save time and maybe energy for complicated paths with many turns.