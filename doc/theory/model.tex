\section{UAV Model}
\label{sec:model}

In this thesis the linearized UAV model presented by Beard \& McLain \cite{uavBEARD} will be used as the prediction model for the path planner that is presented in chapter \ref{ch:optimization}. In this section the UAV states used in this model will be presented, as well as the linearization method.

\subsection{UAV States}
% SNAME should probably be mentioned here
The position of the UAV will be given using the North East Down (NED) coordinate frame denoted $\{n\}$:

\begin{equation}
	\mathbf{p}_{b/n}^n =
	\begin{bmatrix}
		p_N \\ p_E \\ p_D
	\end{bmatrix}
	=
	\begin{bmatrix}
		x_n \\ y_n \\ z_n
	\end{bmatrix},
\end{equation}

while the global location will be given using the Earth Center Earth Fixed (ECEF) frame, denoted $\{e\}$, represented by longitude and latitude:

\begin{equation}
	\bm{\Theta}_{en} =
	\begin{bmatrix}
		l \\ \mu
	\end{bmatrix}.
\end{equation} 

Following the notation used in \cite{uavBEARD}, the velocities of the UAV will be given in the body frame denoted $\{b\}$:

\begin{equation}
	\mathbf{V}^b_g =
	\begin{bmatrix}
		u \\ v \\ w
	\end{bmatrix}.
\end{equation}

The attitude $\bm{\Theta}_{nb}$ of the UAV will be given as Euler-angles, with the corresponding angular velocities $\bm{\dot{\Theta}}_{nb}$:

\begin{equation}
	\bm{\Theta}_{nb} =
	\begin{bmatrix}
		\phi \\ \theta \\ \psi
	\end{bmatrix},
	\hspace{5pt}
	\dot{\bm{\Theta}}_{nb} =
	\begin{bmatrix}
		p \\ q \\ r
	\end{bmatrix}.
\end{equation}

Euler angles are used over quaternions in this paper because the optimization is to be run offline so that computation time is not a critical measure. Even though Euler angles do suffer from gimbal lock while quaternions don't \cite{uavBEARD}, the UAV will not be performing any high-angle maneuvers so that a gimbal lock should never occur.

The UAV model has four control inputs, namely the elevator, aileron, rudder, and throttle:

\begin{equation}
	\mathbf{u} =
	\begin{bmatrix}
		\delta_e \hspace{5pt} \delta_a \hspace{5pt} \delta_r \hspace{5pt} \delta_t
	\end{bmatrix}^\intercal .
\end{equation}


\subsection{Linearizing the Model}

In order to ease computational load the \textit{linear decoupled model} of a UAV, presented by Beard \& McLain \cite{uavBEARD}, will be used in this thesis. While a linear model is not able to fully describe the motions of an aircraft, it is valid around the \textit{trimmed state} of the aircraft. An aircraft in its trimmed state will be able to maintain a straight level flight without any change in the control input, and since the UAV in this thesis is not expected to perform any high-angle maneuvers that puts it far away from the trimmed state, the linear model will be valid. An aircraft in the trimmed state satisfies the following equation:

\begin{equation}
	\mathbf{\dot{x}} = f(\mathbf{x}^*, \mathbf{u}^*) = 0.
\end{equation}

Linearization is performed by adding perturbations to the trimmed state solution, and the linearized states $\mathbf{\bar{x}}$ represent the perturbations away from the trimmed state \cite{modsimEGELAND}. The linearized state is defined as $\mathbf{\bar{x}} = \mathbf{x} - \mathbf{x}^*$, where $\mathbf{x}^*$ is the trimmed state.

The states of an aircraft are highly \textit{coupled}, meaning that they affect each other. This greatly increases the complexity of finding an optimal solution of the model since a change in one variable has effect on more than one state. For this reason the model is also decoupled into lateral and longitudinal models, where the states in one of the models does not affect the states in the other model. This simplification is done by removing terms that has a very small effect on the state, as these effects are easily controlled by the control systems \cite{uavBEARD}. The lateral and longitudinal states are given as:

\begin{equation}
\begin{split}
	&\mathbf{\dot{x}}_{lat} = 
	\begin{bmatrix}
		v \hspace{5pt} p \hspace{5pt} r \hspace{5pt} \phi \hspace{5pt} \psi
	\end{bmatrix} ^\intercal
	, \mathbf{u}_{lat} =
	\begin{bmatrix}
		\delta_a \hspace{5pt} \delta_r
	\end{bmatrix}^\intercal \\
	&\mathbf{\dot{x}}_{lon} =
	\begin{bmatrix}
		u \hspace{5pt} w \hspace{5pt} q \hspace{5pt} \theta \hspace{5pt} h
	\end{bmatrix}^\intercal
	, \mathbf{u}_{lon} =
	\begin{bmatrix}
		\delta_e \hspace{5pt} \delta_t
	\end{bmatrix}^\intercal.
\end{split}
\end{equation}


%\section{Wind and Airspeed}
%Wind will be introduced to the vehicle in order to test how the system withstands disturbances. Since the camera footprint is dependent only on the attitude angles of the aircraft and not the course of the aircraft, the wind will only affect the navigation of the UAV. Wind speed is given in the $\{n\}$ frame as \cite{uavBEARD}

%\begin{equation}
%	\mathbf{V}^n_w =
%	\begin{bmatrix}
%		w_n \\ w_e \\ w_d
%	\end{bmatrix},
%\end{equation}

%and the air speed $\mathbf{V}_a$ of the aircraft is given by the wind speed $\mathbf{V}_w$ and ground speed $\mathbf{V}_g$ as

%\begin{equation}
%\begin{split}
%	\mathbf{V}_a & = \mathbf{V}_g - \mathbf{V}_w\\
%	\begin{bmatrix}
%		u_r \\ v_r \\ w_r
%	\end{bmatrix}
%	& =
%	\begin{bmatrix}
%		u \\ v \\ w
%	\end{bmatrix}
%	- \mathbf{R}_n^b
%	\begin{bmatrix}
%		w_n \\ w_e \\ w_d
%	\end{bmatrix}
%\end{split}
%\end{equation}

%where $\mathbf{R}_n^b$ is the rotation matrix between the NED frame $\{n\}$ and the body frame $\{b\}$. 

%% Is the last part really needed?

%When in the presence of wind, the heading of the UAV isn't necessarily the direction that the UAV is moving. Wind will introduce a crab angle $\chi_c$ that together with the heading $\psi$ gives the course angle $\chi$:
%
%\begin{equation}
%	\chi = \psi + \chi_c.
%\end{equation}
