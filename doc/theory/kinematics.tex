\section{Kinematics}

When the camera is fixed to the aircraft body, what is actually captured by the camera, the camera footprint, depends on the angles of the aircraft. The equations needed to express the position of the centre and outer points of the camera footprint will be given here. The kinematic model for the UAV used in this paper will also be given here.


\subsection{UAV States}

The position of the UAV will be given using the North East Down (NED) coordinate frame, denoted $\{n\}$:

\begin{equation}
	\mathbf{p}_{b/n}^n =
	\begin{bmatrix}
		N \\ E \\ D
	\end{bmatrix}
	=
	\begin{bmatrix}
		x_n \\ y_n \\ z_n
	\end{bmatrix}.
\end{equation}

The attitude of the UAV will be given as Euler-angles:

\begin{equation}
	\bm{\Theta}_{nb} =
	\begin{bmatrix}
		\phi \\ \theta \\ \psi
	\end{bmatrix}	
\end{equation}

with corresponding angular velocities:

\begin{equation}
	\dot{\bm{\Theta}}_{nb} =
	\begin{bmatrix}
		p \\ q \\ r
	\end{bmatrix}.
\end{equation}

ADD WIND KINEMATICS


\subsection{Camera Footprint}

The camera footprint is coupled with all of the three angles given in $\bm{\Theta}$. The position of the camera footprint will be calculated using forward kinematics, and the "SITUATION" is shown in figure \ref{fig:footprint_centre}.

\begin{figure}
	\import{/}{kinematics_footprint_figure.tex}
	\caption{Illustration of how the aircraft attitude influence the camera position.}
	\label{fig:footprint_centre}
\end{figure}

\subsubsection{Centre Position}

The attitude of the UAV is given in the body frame $\{b\}$ and the height $z_n$ is given in the NED frame $\{n\}$, and the model assumes flat earth. The position of the footprint centre point $\mathbf{c}_b^b$ in the body frame $\{b\}$ is expressed as the geometric (???) distance from the UAV position to the footprint centre point:

\begin{equation}
	\label{eq:camera_position_body}
	\mathbf{c}_b^b =
	\begin{bmatrix}
		c_{x/b}^b \\ c_{y/b}^b
	\end{bmatrix}
	=
	\begin{bmatrix}
		z_n tan(\theta) \\ z_n tan(\phi)
	\end{bmatrix}.
\end{equation}

The coordinates of the camera position in $\{n\}$ can be found by rotating the point $\mathbf{c}_b^b$ with respect to the aircraft heading $\psi$, and by translating the rotated point to the aircrafts position in the $\{n\}$ frame. The rotation matrix for rotating with respect to the heading is given as

\begin{equation}
	\mathbf{R}_{z,\psi} =
	\begin{bmatrix}
		cos(\psi) & -sin(\psi) \\
		sin(\psi) & cos(\psi)
	\end{bmatrix}.
\end{equation}

The final expression for the camera footprint centre position $\mathbf{c}^n$ in the $\{n\}$ frame then becomes:

\begin{equation}
\label{eq:camera_position_ned}
\begin{split}
	\mathbf{c}^n & = \mathbf{p} + \mathbf{R}_{z,\psi} \mathbf{c}_b^b \\
	& =
	\begin{bmatrix}
		x_n \\ y_n
	\end{bmatrix}
	+ \mathbf{R}_{z,\psi}
	\begin{bmatrix}
		x_{x/b}^b \\ c_{y/b}^b
	\end{bmatrix}
\end{split}
\end{equation}


\subsubsection{Edge Points}

A hyperspectral pushbroom sensor captures images in a line, and the centre point of the camera footprint does not express the entire area that is captured by the sensor. The edge points of the camera footprint are calculated with respect to the sensor's field of view, as shown in figure \ref{fig:kinematics_edge_points}. These points $\mathbf{e}$ can be found by altering \ref{eq:camera_position_body}:

\begin{equation}
	\mathbf{e}_{1,b}^b =
	\begin{bmatrix}
		z_n tan(\theta) \\ z_n tan(\phi + \sigma)
	\end{bmatrix}
	, \hspace{5pt}
	\mathbf{e}_{2,b}^b =
	\begin{bmatrix}
		z_n tan(\theta) \\ z_n tan(\phi - \sigma)
	\end{bmatrix}.
\end{equation}

\begin{figure}
	\import{/}{kinematic_edge_points_figure.tex}
	\caption{Illustration of how the field of view for a pushbroom sensor is calculated.}
	\label{fig:kinematics_edge_points}
\end{figure}

The steps for writing the edge points $\mathbf{e}$ in the $\{n\}$ is similar as in equation \ref{eq:camera_position_ned}:

\begin{equation}
	\mathbf{e}^n = \mathbf{p} + \mathbf{R}_{z,\psi} \mathbf{e}_b^b.
\end{equation}