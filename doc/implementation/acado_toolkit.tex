\section{ACADO toolkit}

The ACADO Toolkit \cite{acadoHOUSKA} is an open-source toolkit that supports several different methods for solving optimization problems. The toolkit provides methods to solve four different classes of optimization problem: Optimal control problems, multi-objective optimization and optimal control problems, parameter and state estimation problems, and model predictive control.

Even though the toolkit will be used to create an MPC in this paper, the optimal control problems (OCP) class will be used to solve the optimization problem. The reason for this is that between each iteration of the MPC algorithm a new trajectory must be generated, and the MPC problem class does not have the functionality needed to do this.

%To solve the optimization problems the toolkit uses many different algorithms. It also has its own Runge-Kutta and BDF integrators to simulate both ODE's (Ordinary Differential Equation) and DAE's (Differential Algebraic Equation). A MATLAB interface is also supplied by the toolkit, but this will not be used for this thesis.


\subsection{Runge-Kutta Method}

The Runge-Kutta method is a form of \textit{numerical integrator} that can be used to solve differential equations, and is used by the ACADO toolkit to integrate the prediction model. The method is based on the Euler method, which is a very simple method for numerical integration.

The ACADO toolkit provides algorithms for \textit{explicit} Runge-Kutta methods \cite{acadoHOUSKA}, where explicit means that the method calculates the state of the system at a later time based on the current state. The Runge-Kutta method calculates the later state of the system by calculating several approximations of the derivative of the system. The current state of the system, together with a linear combination of the approximated derivatives, gives the next state of the system \cite{modsimEGELAND}. For a system on the form $\mathbf{\dot{y}} = \mathbf{f}(\mathbf{y},t)$, the Runge-Kutta method can be mathematically expressed as:

\begin{subequations}
\begin{equation}
	\mathbf{k}_i = \mathbf{f}(\mathbf{y}_n + h \sum_{j=1}^{i-1} a_{ij}\mathbf{k}_j, t_n + c_ih), i = 1, ..., \sigma
\end{equation}
\begin{equation}
	\mathbf{y}_{n+1} = \mathbf{y}_n + h \sum_{j=1}^\sigma b_j \mathbf{k}_j
\end{equation}
\end{subequations}

where $t_n$ is the current time, $\mathbf{y}_n$ is the output of the system at the current time, $h$ is the step size and $\sigma$ is the number of approximations of the derivative that is calculated. $a$, $b$ and $c$ are parameters for the specific Runge-Kutta method used, and $c$ must satisfy $0 < c < 1$.

Systems may consist of dynamics of different time constants, and the difference between the constants may have a big impact on how fast an explicit method can perform the computations. While they don't have a formal definition, systems that explicit methods compute poorly are referred to as \textit{stiff systems} \cite{stiffBUI}. The poor performance of explicit method comes from the stability of these methods, as the stability is dependent on the step size. This means that for systems with dynamics of varying time constants, a very small stepsize has to be chosen in order for the method to remain stable. For stiff systems \textit{implicit} methods have a better performance and faster computation time. 

%\subsection{Solver}

%In order to solve the optimization problem, ACADO uses some kind of solver.

%\subsubsection{KKT-Tolerance}

%In order to rate the "goodness" of a solution, ACADO uses some kind of KKT-Tolerance.