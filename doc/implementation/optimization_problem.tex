\section[Optimization Problem]{Implementing the Optimization Problem}
%\sectionmark{Optimization Problem}

Since ACADO offers a symbolic way of implementing the optimization problem the equations from chapter \ref{ch:optimization} can be implemented as they stand. The 12 UAV states $\mathbf{x}$ are implemented as differential states, as well as the four control surfaces $\mathbf{u}$. The control rate $\mathbf{\dot{u}}$ is defined as the control states of the problem, and are linked to the control states through (\ref{eq:control_relation}). The model of the UAV is implemented using the differential equations for each state.

ACADO offers a symbolic way of defining the objective function as an LSQ as well. By defining what states are included in the LSQ and their weights, ACADO automatically minimizes the objective function. The reference model can either be a constant value or a time varying variable. When a path is to be tracked, the path needs to be given with related timepoints.

Lastly, the constraints are also implemented using the symbolic syntax, by simply assigning max-min values for the variables that are subject to constraints. The initial value of the states is also set using the same syntax. 


\subsection{Nonlinear Prediction Model}

Initially, effort was made to implement the nonlinear model presented by Beard \& McLain \cite{uavBEARD} as the prediction model in the optimization problem. This would have given more precise results as the nonlinear model is a closer representation of the real UAV. Since the nonlinear model also includes the effect wind has on the aircraft, the path could be optimized with the knowledge about the wind conditions as well. The level of calculation needed for the nonlinear model is significantly higher; however, since this implementation is intended to run offline before the flight occurs, computation time is not a critical concern.

Achieving stable flight within the optimization problem with the nonlinear model on the other hand, turned out to be a difficult task that was far from trivial. This is somewhat due to the nonlinearity, but also to the high coupling between states in the model. The coupling causes changes in one state to affect many other states, which results in a much more complex problem. Several different algorithm and solver settings in ACADO was tested, as well as different objective functions and weighting of these functions. After many attempts the decision to use the linear model instead was made, largely due to this being a project with limited time available.


\subsection{Nonlinearity}

The ACADO toolkit is written for nonlinear optimization problems, and using it together with a linear optimization will give a correct solution, but the computation time for a linear problem will be longer than it needs to be because of extra overhead related to nonlinear algorithms.

For this reason, using it with a linear prediction model may seem odd. However, the cost function used in this problem is not linear. This is because of both the calculations for the position $\mathbf{p}_N$ and $\mathbf{p}_E$ is represented by nonlinear equations, and the equations to calculate the camera footprint are nonlinear. If the timing demands for this optimization problem was of importance, a solution may be to linearize the position equations as well as the equations used to calculate the camera footprint, and then implement the optimization problem using a toolkit that is made for linear problems.