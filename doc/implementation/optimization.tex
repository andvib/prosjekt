\chapter{MPC Implementation}

The offline intervalwise MPC presented in chapter \ref{ch:optimization} will be implemented using C++. It will consist of two main parts: the MPC algorithm and the optimization solver. The optimization solver will be implemented using the ACADO Toolkit \cite{acadoHOUSKA}.


\section{MPC}

The task of the MPC part of the implementation is to supply the ACADO implementation with the information needed to perform the optimization, and also control the optimization algorithm so that the correct horizon is calculated, as well as storing the results in the correct order. The pseudocode for the MPC implementation is shown in algorithm \ref{alg:mpc} and \ref{alg:genHor}.

\begin{algorithm}
\caption{Offline Intervalwise MPC Algorithm}
\label{alg:mpc}
\begin{algorithmic}
\Procedure{MPC}{}
	\State $\textit{path} \gets \text{path from file}$
	\State $\textit{timestep} \gets \text{duration of timestep}$
	\State $\textit{horizonlen} \gets \text{number of } \textit{timestep} \text{ in horizon}$
	\State $\textit{intervallen} \gets \text{number of } \textit{timestep} \text{ in interval}$
	\State $\textit{no intervals} \gets \text{number of } \textit{interval} \text{ needed to cover } \textit{path}$
	\State $\textit{x0} \gets \text{initial values of differential states}$
	\State $\textit{u0} \gets \text{initial values of control states}$
	\State $\textit{du0} \gets \text{inital values of control rates}$
	\For{each \textit{no intervals}}
		\State $\textit{horizon} \gets \Call{GenerateHorizon}{path, timestep, intervallen, pathlen, x_{pos}, y_{pos}}$
		\State Solve optimization with initial states \textit{x0 u0 du0} for current \textit{horizon}
		\State $\textit{x0} \gets \text{ last } \textit{x} \text{ value in the interval}$
		\State $\textit{u0} \gets \text{ last } \textit{u} \text{ value in the interval}$
		\State $\textit{du0} \gets \text{ last } \textit{du} \text{ value in the interval}$
		\State Save results for \textit{interval}
	\EndFor
\EndProcedure
\end{algorithmic}
\end{algorithm}

\begin{algorithm}
\caption{Generate horizon}
\label{alg:genHor}
\begin{algorithmic}
\Procedure{GenerateHorizon}{\textit{path timestep horizonlen pathlen $x_{pos}$ $y_{pos}$ speed}}
	\State $\textit{no timesteps} \gets \textit{horizonlen}/\textit{timestep}$
	\State $\textit{distance} \gets \textit{speed} * \textit{timestep}$
	\For{each \textit{no timesteps}}
		Find point on \textit{path} with the given \textit{distance} away from current \textit{$x_{pos}$} and \textit{$y_{pos}$}
		
	\EndFor
\EndProcedure
\end{algorithmic}
\end{algorithm}


\section{ACADO toolkit}

The ACADO Toolkit is an open-source toolkit that supports several different methods for solving optimization problems. The toolkit provides four problem classes that is can solve: Optimal control problems, multi-objective optimization and optimal control problems, parameter and state estimation problems, and model predictive control. Even though this thesis is to solve an MPC problem, the optimal control problem class will be used. This is because the reference trajectory have to be calculated between each step of the MPC, which is easier done outside of the ACADO toolkit.

To solve the optimization problems the toolkit uses many different algorithms. It also has its own Runge-Kutta and BDF integrators to simulate both ODE's (Ordinary Differential Equation) and DAE's (Differential Algebraic Equation). A MATLAB interface is also supplied by the toolkit, but this will not be used for this thesis.


\subsection{Algorithms}

In this chapter the most important algorithms that are used by the ACADO Toolkit to solve the optimization problem will be briefly presented.


\subsection{Runge-Kutta Integrator}
