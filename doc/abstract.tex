\section*{Abstract}

Fixed-wing Unmanned Aerial Vehicles are extensively used for ground observation as they allow for effective and precise ways for collecting information, but it has a downside with the quality of the information being dependent on the attitude of the aircraft. There is a coupling between the sensors and the aircraft's attitude states, which is usually decoupled by mounting the sensors to a gimbal which increases the weight of the payload. This paper investigates alternative control methods that minimizes the errors caused by the aircraft attitude states when sensors are fixed to the aircraft body. The control methods will be developed with the usage of a pushbroom hyperspectral camera in mind, and they will be simulated in Matlab on a model of the Aerosonde UAV. A course PD-controller using the rudder will be derived and simulated as a way of reducing roll angle during flight. A path planner that alters a predetermined ground track that is to be surveyed based on knowledge about the roll used when changing course will also be developed. The results show that the controller using rudder to correct the course significantly reduces roll and the corresponding lateral movement of the camera footprint. Some roll is still present, and the slower response of the rudder controller constraints the degree of curves in the path. The path generated by the path planner results in smoother roll angles, and the path stays within the camera's field of view through turns. When following the generated path the aircraft still causes lateral movement of the camera footprint which causes the points of interest to move away from the center of the camera's field of view. The methods developed here show that it is possible to reduce image errors caused by the UAVs attitude states, but neither of the methods are able to guarantee that the points of interest are covered.