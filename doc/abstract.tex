\section*{Abstract}

Fixed-wing Unmanned Aerial Vehicles are exensively used for ground observation as the allow for effective and precise ways for collecting information, and several types of sensors can be used. When sensors are attached to the aircraft's body there is a coupling between the aircraft's attitude states and the what the sensors measures, which again influence the quality of the data acquired. Today this problem is solved by attaching the sensors to a gimbal mounted to the aircraft's body, which decouples the sensor from the attitude states. This leads to increased weight of the system and may alter the aerodynamics of the aircraft. This paper investigates alternative control methods that minimizes the errors caused by the aircraft attitude states when sensors are fixed to the aircraft body. The control methods will be developed with the usage of a pushbroom hyperspectral camera in mind, and they will be simulated in Matlab on a model of the Aerosonde UAV. A course PD-controller using the rudder to alter course instead of ailerons will be derived and simulated as a way of reducing roll used during flight. A path planner that alters a predetermined ground track that is to be surveyed based on knowledge about the roll used when following the track will also be developed. This path planner allows for roll during operation, wihtout losing the points of interest from the camera's field of view. The results show that the controller using rudder to correct the course significantly reduces roll and the corresponding lateral movement of the camera footprint compared to a controller using ailerons to correct course. However some roll is still present, and the slower response of the rudder controller constraints the degree of curves in the path. The path generated by the path planner is a smoother path which gives smoother rolling of the aircraft, and the path stays within the camera's field of view through turns. When following the generated path the aircraft still causes latera movement of the camera footprint, which causes the points of interest to not be in the center of the cameras field of view, and the method has no guarantee of keeping the points of interests centered. The methods developed here show that it is possible to reduce image errors caused by the UAVs attitude states, but neither of the methods are able to guarantee that the points of interest are surveyed.