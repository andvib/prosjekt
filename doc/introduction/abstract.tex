\chapter*{Abstract}

Fixed-wing Unmanned Aerial Vehicles are extensively used for ground observation as they allow for effective and precise methods of collecting information, but they have a downside as the quality of the information gathered depends on the attitude of the aircraft. There is a coupling between the sensors and the aircrafts attitude states, which is usually decoupled by mounting the sensors to a gimbal which increases the weight of the payload. This paper investigates a control method that minimizes the error caused by the aircraft attitude states when a hyperspectral pushbroom camera is fixed to the aircraft body. This will be done by developing an offline intervalwise Model Predictive Control algorithm that seeks to minimize the distance between the camera centre point and a pre-defined ground path that is to be observed. Since this is an offline method, it is to be run before the flight is initiated. The algorithm is implemented using C++ and the ACADO Toolkit, and the optimized paths will be simulated using a Software-in-the-Loop simulator. The application developed is able to optimize both curved and piecewise linear paths. It is able to optimize linear corners with an angle of $45\degree$, and curved turns with an angle of $90\degree$ and also $180\degree$. However, testing proved that the application was not very robust when it comes to optimizing paths containing several sharp turns quickly after another, or long straight paths right after a turn. In addition the optimization is very time consuming. Simulation of the optimized path show that the precision of tracking the path can be done with a mean error of below $5$m.