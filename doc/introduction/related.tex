\section{Related Work}

The most common method to decouple the UAV attitude states from the sensor today is to equip the aircraft with a gimbal, which results in easy UAV operation without losing track of the features that is to be observed. Since the gimbal angles have limited range, features can be lost from the camera field of view (FOV) for some combinations of aircraft positions and attitudes. Previous solutions to this problem include generating trajectories that ensures that the gimbal angles are able to cover the features of interest \cite{nundalSKJONG}, or putting constraints on the roll angle and altitude of the UAV \cite{constraintsEGBERT}.

A simpler solution to avoid lateral movements of the FOV is to change the UAV course by using the rudder instead of the ailerons. The rudder deflection creates a yawing moment which causes the aircraft to change course \cite{ratcFISHER}. This type of controller is referred to both as a Rudder Augmented Trajectory Correction (RATC) controller \cite{ratcFISHER} and a skid-to-turn (STT) controller \cite{skidMILLS}. Results show that the performance of these controllers are comparable to conventional controllers using roll to change course, and that errors in the images is greatly reduced \cite{ratcFISHER}\cite{skidMILLS}\cite{alternateAHSAN}.

While the controllers offer a solution to the control problem that reduces the errors in the images, they do not ensure that the features of interest stay inside the sensors FOV. To ensure that they stay within the FOV, Jackson has developed an optimization algorithm that minimizes the error of the sensor footprint \cite{optimJACKSON}. Jakcson's solution differs from the solution proposed in this paper as his solution uses a simplified model of the UAV to calculate the optimal path online, while this paper focuses on using a more precise model to find an optimal solution before the flight is initiated.

Jackson presents a path planner that aims to minimize the error between the target on the ground, and the footprint of a camera fixed to a UAV by using a Nonlinear Model Predictive Controller (NMPC). The NMPC is compared to a PID and a sliding-mode controller that seek to follow the same path. Simulations of the three controllers show that of the three, the PID controller had the biggest crosstrack error. The results of the simulation of the NMPC and the sliding-mode controller showed that the two had comparable performance. The NMPC controller was able to find a near optimal solution with the performance characteristics of a real-time application.

One important point made by Jackson is that perfect tracking of a ground path with a fixed camera is not possible when using the roll angle to change the course of the aircraft. A controller that attempts to solve this problem would be unstable because of the dependence the camera position has on the roll angle. When the path turns right and the aircraft rolls right to follow this path, the camera position will move left, away from the path. This only applies to controllers that attempt a perfect tracking of the path, so that a near-perfect tracking is still achievable.