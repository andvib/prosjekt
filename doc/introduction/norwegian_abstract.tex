\chapter*{Sammendrag}

Ubemannede fly blir ofte brukt til å observere strukturer på bakken, siden de på en enkel og presis måte kan samle inn informasjon ovenfra. De har derimot en ulempe ved at kvaliteten på informasjonen som samles inn avhenger av orienteringen til flyet, siden det er en kobling mellom sensoren og flyets vinkler. Vanligvis håndteres denne koblingen ved å feste sensorene til en stabilisator som motvirker effekten av vinklene til flyet, men denne løsningen medfører ekstra vekt i nyttelasten til flyet. Denne avhandlingen ser på en kontrollmetode som minimerer negative effekter av denne koblingen når et hyperspektralt linje-kamera som er festet direkte til flykroppen blir brukt til bakke-observering. Dette gjøres ved å utvikle en frakoblet intervallvis modell-prediktiv regulator som forsøker å minimere avstanden mellom midtpunktet til kameraet på bakken, og til ruten på bakken som skal observeres. Siden dette er en frakoblet kontrollmetode kjøres den før flygningen starter. Algoritmen blir implementert i C++ med ACADO Toolkit, og den optimaliserte ruten vil bli simulert ved hjelp av en "software-in-the-loop" simulator. Applikasjonen kan optimalisere både kurvede og lineært stykkvise ruter. Den er i stand til å optimalisere kurvede $90\degree$ svinger med en radius på $50$m, og $180\degree$ svinger med en radius på minst $200$m. For lineære hjørner er det skarpeste den klarer å optimalisere $45\degree$. Hvis ruten består av flere svinger er ikke applikasjonen robust nok til å optimalisere ruten om svingene er for nærme hverandre. Samtidig håndterer den lange rette strekker mellom svinger dårlig, siden den ikke posisjoner flyet rett over ruten som skal observeres. Applikasjonen bruker veldig lang tid på å optimalisere ruter, et sekund med flyvning kan ta $150$s å optimalisere. Simuleringer av den optimaliserte ruten viser at ruten på bakken kan observeres med en presisjon på under $5$m, og den optimaliserte ruten for lineære ruter er litt mer presis en for kurvede.