\chapter{Conclusion}


\section{Future Work}

Testing shows that one of the major shortcomings of the MPC developed in this thesis is how the trajectory is generated for each iteration of the MPC. The method currently used is not precise enough, which leads to spikes in the reference used to solve the optimization problem. If the trajectory generation is improved it would hopefully remove many of the oscillations seen in these results, and make it possible to optimize even sharper corners. A more sophisticated trajectory generator could also allow the MPC to support loitering above a point.

Another problem area for the MPC that should be improved is the duration it takes to optimize the paths. Even though this application isn't intended to run in realtime, it uses what should be an unnecessary amount of time. The ACADO Toolkit offers a code-generation tool, that generates C++ code to be run instead of "re-compiling" the optimization problem for every iteration of the MPC. Using the code-generation tool could improve the time problem.

Since the MPC already uses a linear model of the UAV, the cost function could also be linearized in order to get a problem that takes less time to solve. Since the ACADO Toolkit is not optimal for linear problems, the linearized problem should be implemented with a different toolkit intended for solving linear problems.

In order to achieve more precise optimization, the complete nonlinear model of the UAV should be implemented. It was attempted in this thesis, but eventually given up on because of time restriction. With more time this should be fully possible to achieve. With the complete nonlinear model it is also possible to include wind in the optimization problem. And lastly, it should be explored including changing altitudes in the optimization problem.