\section{Controller}

In this chapter the dynamic equations needed to describe how the course of the aircraft changes with the rudder will be presented, and the transfer function for the controller is derived.

\subsection{Dynamics}
In his paper, Fisher \cite{ratcFISHER} uses the dynamic model given in Beard and McLain \cite{suaBEARD} to develop a controller that uses the rudder to change the heading. A similar controller will be used in this paper, and it will be derived by using the same steps that Fisher used.

To simplify the controller, it will be assumed that there is no wind and no sideslip $\beta$. These assumption will simplify the control problem since it can be assumed that $\chi = \psi$. It will also be assumed that the UAV is in trimmed, straight level flight, as this will simplify the system since the roll angle $\phi$ and pitch angle $\theta$ both can be assumed to be small.

The yaw dynamics for a UAV are (eq. 3.17, Beard and McLain \cite{suaBEARD})

\begin{equation}
	\dot{r} = \Gamma_7pq - \Gamma_1qr + \Gamma_4l + \Gamma_8n
	\label{eq:yaw_dynamics}
\end{equation}

where $l$ and $n$ are the moments about the $\bm{i}^b$ and $\bm{j}^b$ axes respectively. The $\Gamma$ equations describe the inertia of the aircraft and are expressed using elements of the inertia matrix $\bm{J}$.

The heading dynamic is expressed by the pitch rate $q$, the yaw rate $r$, and the attitude states $\bm{\Theta}_{nb}$ (eq. 3.3, Beard and McLain \cite{suaBEARD}):

\begin{equation}
	\dot{\psi} = sin(\phi)sec(\theta)q + cos(\phi)sec(\theta)r.
\end{equation}

As mentioned it is assumed that the aircraft is in trimmed straight level flight. By assuming small $\phi$ and $\theta$, and also no pitch rate $q$, the heading dynamics can be simplified to:

\begin{equation}
	\dot{\psi} = r,
\end{equation}

which leads to:

\begin{equation}
	\ddot{\psi} = \dot{r}.
\end{equation}

The equation for the yaw dynamics (\ref{eq:yaw_dynamics}) can now be written as

\begin{equation}
	\label{eq:yaw_dynamics2}
	\ddot{\psi} = \dot{r} = \Gamma_4l + \Gamma_8n.
\end{equation}

The moments $l$ and $n$ are the moments on the aircraft caused by the attitude states and rates, the sideslip $\beta$, and also the aileron deflection $\delta_a$ and the rudder deflection $\delta_r$. These are given by equation 4.15 and 4.16 in Beard and McLain \cite{suaBEARD}:

\begin{subequations}
\begin{equation}
	\label{eq:controller_moments}
	l = \frac{1}{2} \rho V_a^2Sb[C_{l_0} + C_{l_\beta}\beta + C_{l_p}\frac{b}{2V_a}p + C_{l_r}\frac{b}{2V_a}r + C_{l_{\delta_a}}\delta_a + C_{l_{\delta_r}}\delta_r]
\end{equation}
\begin{equation}
	n = \frac{1}{2} \rho V_a^2Sb[C_{n_0} + C_{n_\beta}\beta + C_{n_p}\frac{b}{2V_a}p + C_{n_r}\frac{b}{2V_a}r + C_{n_{\delta_a}}\delta_a + C_{n_{\delta_r}}\delta_r].
\end{equation}
\end{subequations}

By continuing to follow Fishers \cite{ratcFISHER} notation, equations (\ref{eq:yaw_dynamics2}) and (\ref{eq:controller_moments}) can be combined to

\begin{equation}
	\label{eq:yaw_dynamics_final}
	\ddot{\psi} = \frac{1}{2}V_a^2Sb[C_{r_0} + C_{r_\beta}\beta + C_{r_p}\frac{b}{2V_a}p + C_{r_r}\frac{b}{2V_a}r + C_{r_{\delta_a}}\delta_a + C_{r_{\delta_r}}\delta_r]
\end{equation}

where

\begin{subequations}
\begin{equation}
	C_{r_0} = \Gamma_4C_{l_0} + \Gamma_8C_{n_0}
\end{equation}
\begin{equation}
	C_{r_\beta} = \Gamma_4C_{l_\beta} + \Gamma_8C_{n_\beta}
\end{equation}
\begin{equation}
	C_{r_p} = \Gamma_4C_{l_p} + \Gamma_8C_{n_p}
\end{equation}
\begin{equation}
	C_{r_r} = \Gamma_4C_{l_r} + \Gamma_8C_{n_r}
\end{equation}
\begin{equation}
	C_{r_{\delta_a}} = \Gamma_4C_{l_{\delta_a}} + \Gamma_8C_{n_{\delta_a}}
\end{equation}
\begin{equation}
	C_{r_{\delta_r}} = \Gamma_4C_{l_{\delta_r}} + \Gamma_8C_{n_{\delta_r}}
\end{equation}
\end{subequations}

where the constants are craft-specific parameters, and

\begin{subequations}
\begin{equation}
	\Gamma_4 = \frac{J_{xz}}{J_xJ_z-J_{xz}^2}
\end{equation}
\begin{equation}
	\Gamma_8 = \frac{J_x}{J_xJ_z-J_{xz}^2}.
\end{equation}
\end{subequations}


\subsection{Controller Transfer Function}
Since the controller is to use rudder input $\delta_r$ to alter the heading $\psi$, equation (\ref{eq:yaw_dynamics_final}) can be rearranged to express these variables:

\begin{equation}
	\label{eq:yaw_control}
	\ddot{\psi} = -a_{\psi_1}\dot{\psi} + a_{\psi_2}\delta_r + d_\psi
\end{equation}

where

\begin{subequations}
\begin{equation}
	a_{\psi_1} = -\frac{1}{4}\rho V_aSb^2C_{r_r}
\end{equation}
\begin{equation}
	a_{\psi_2} = \frac{1}{2}\rho V_a^2Sb^2C_{r_{\delta_r}}
\end{equation}
\begin{equation}\label{eq:constants_controller}
	d_\psi = \frac{1}{2}\rho V_a^2Sb[C_{r_0} + C_{r_\beta}\beta + C_{r_p}\frac{b}{2V_a}p + C_{r_{\delta_a}}\delta_a].
\end{equation}
\end{subequations}

$a_{\psi_1}$ is chosen to be negative as this will ease later calculations (see (\ref{eq:controller_kp})). The Laplace transformation brings (\ref{eq:yaw_control}) to the form

\begin{equation}
	\label{eq:yaw_controller_s}
	\psi(s) = \frac{a_{\psi_2}}{s(s+a_{\psi_1})}\delta_r(s) + \frac{1}{s(s+a_{\psi_1})}d_\psi(s).
\end{equation}

This equation show that the second term containing $d_\psi$ acts as a disturbance for the controller. As shown in (\ref{eq:constants_controller}), the inputs to this term are the sideslip $\beta$, roll rate $p$, and aileron deflection $\delta_a$. Since the UAV is assumed to be in trimmed straight level flight and the controller will use the rudder to turn instead of roll it is already assumed that the roll rate $p$ will be zero, as will the aileron deflection $\delta_a$. During normal operation it cannot be assumed that no sideslip $\beta$ will occur. However, any $\beta$ is assumed to be small so that it can be removed from the controller equation. The final transfer function for the controller dynamics will then be

\begin{equation}
	\label{eq:tf_rudder}
	\frac{\psi(s)}{\delta_r(s)} = \frac{a_{\psi_2}}{s(s+a_{\psi_1})}.
\end{equation}

In order to control the heading of the UAV with the help of the rudder, a controller must be added. The PD controller used here takes the form

\begin{equation}
	\delta_r = ek_p + \dot{e}k_d
\end{equation}

where $e$ is defined as the error between the desired heading $\psi_d$ and the measured heading $\psi$

\begin{equation}
	e = \psi_d - \psi.
\end{equation}

The transfer function between the desired heading and the measured heading is found by adding the controller to the transfer function between rudder and heading (\ref{eq:tf_rudder})

\begin{equation}
	\label{eq:rudder_PD}
	\frac{\psi}{\psi_d} = \frac{a_{\psi_2}k_p}{s^2 + (a_{\psi_1}+a_{\psi_2}k_d)s + a_{\psi_2}k_p}.
\end{equation}

Since the transfer function is written in the form of a canonical second-order transfer function, the proportional gain $k_p$ and the derivative gain $k_d$ can be found by calculating the natural frequency $\omega_n$ and damping factor $\zeta$. The final expressions for the gains will be

\begin{subequations}
\begin{equation}
	\label{eq:controller_kp}
	k_p = \frac{\omega_n^2}{a_{\psi_2}}
\end{equation}
\begin{equation}
	k_d = \frac{2\zeta\omega_n - a_{\psi_1}}{a_{\psi_2}}.
\end{equation}
\end{subequations}

%** PROBABLY NOT TO BE USED... OR IS IT?**
%A PID controller takes the form
%\begin{equation}
%	\label{eq:rudder_PID}
%	\frac{\psi}{\psi_d} = \frac{-a_2(k_p + \frac{1}{s}k_i)s}{s^3 + (a_2K_d-a_1)s^2 - a_2k_ps + a_2k_i}
%\end{equation}
%********************************************