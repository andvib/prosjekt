\section{Simulation of Controller}

For the simulation the controller was implemented in Simulink, and it operates alongside the autopilot described in chapter \ref{ch:simulation}. Since the controller will be used to control course using the rudder, the autopilot will be controlling all the other states and actuators.


\subsection{Controller Implementation}

The controller was implemented using a simple block diagram in Simulink, with desired course as input and rudder control as output. As a starting point for the controller tuning the control loop was simulated in an open loop. 

\subsection{Test Cases}


The altitude used when using a pushbroom sensor from an UAV for ground observation varies with what is being observed and the equipment used. When observing the vegetation, low-altitudes around $100$ m is often used (\cite{hymsySUOMALAINEN}, \cite{wheatLELONG}, \cite{lowRAMIREZ}). However, altitudes as high as $1900$ m has been used to observe agricultural crops \cite{mosaicASMAT}. In this paper simulations will be performed mostly at $100$ m, with some simulations at higher altitudes for comparison. The FOV for the camera will be set to $19\degree$ (approximately the same as in \cite{hymsySUOMALAINEN}).

The controller has been tested in three different cases. The first case is a simple $45\degree$ turn in order to test the step response of the controller. The second case will follow a path in order to compare the controller with a "regular" course controller. The third and last case is the same path as in the second case, but with wind.


\subsection{Results Case 1}


\subsection{Results Case 2}


\subsection{Results Case 3}


\subsection{Results}